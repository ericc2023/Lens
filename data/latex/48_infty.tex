\documentclass[a4paper,12pt]{article}
\usepackage{latexsym}
\usepackage{amsmath}
\usepackage{amssymb}
\usepackage{graphicx}
\usepackage{wrapfig}
\pagestyle{plain}
\usepackage{fancybox}
\usepackage{bm}

\begin{document}

among never-takers with $D = 1$ (as defiers do not exist). Likewise, ?011 corre-

sponds to the direct effect on always-takers with $D=0$. Indeed, the results in Table

suggest that both parameters are consistently estimated with the change-in-changes

model (Panel A.).

1 Proof of Theorem 1

1.1 Average direct effect under $\mathrm{d}=1$ conditional on $\mathrm{D}=1$

and $\mathrm{M}(1)=0$

In the following, we prove that ?11 $0(1) = E[Y_{1}(1,0) -Y_{1}(0,0)|D = 1, M_{i}(1) =$

$0] = E[Y_{1}\ -Q_{00}(Y_{0})|D\ =\ 1,\ M\ =\ 0]$. Using the observational rule, we obtain

$E[Y_{1}(1,0)|D\ =\ 1,\ M(1)\ =\ 0] = E[Y_{1}|D\ =\ 1,\ M\ =\ 0]$. Accordingly, we have to

show that $E[Y_{1}(0,0)|D\ =\ 1,\ M(1)\ =\ 0] = E[Q_{00}(Y_{0})|D\ =\ 1,\ M\ =\ 0]$ to finish the

proof. Denote the inverse of $h(d,\ m,\ t,\ u)$ by $h^{-1}(d,\ m,\ t;y)$ , which exists because of

the strict monotonicity required in Assumption 1. Under Assumptions 1 and $3\mathrm{a}$, the

conditional potential outcome distribution function equals
$$
F_{Y_{\mathrm{t}}(d,0)|D=1,M=0}(y)^{A1}=\mathrm{P}\mathrm{r}(h(d,\ m,\ t,\ U)\ \leq y|D=1,\ M=0,\ T=t)\ ,
$$
$$
=\mathrm{P}\mathrm{r}(U\leq h^{-1}(d,\ m,\ t;y)|D=1,\ M=0,\ T=t)\ ,
$$
(1.1)
$$
A3a=\mathrm{P}\mathrm{r}(U\leq h^{-1}(d,\ m,\ t;y)|D=1,\ M=0)\ ,
$$
$$
=F_{U|10}(h^{-1}(d,\ m,\ t;y))\ ,
$$
for $d, d' \in \{0$, 1$\}$. We use these quantities in the following. First, evaluating

$F_{Y_{1}(0,0)|D=1,M=0}(y)$ at $h(0,0,1,\ u)$ gives
$$
F_{Y_{1}(0,0)|D=1,M=0}(h(0,0,1,\ u))=F_{U|10}(h^{-1}(0,0,1;h(0,0,1,\ u)))=F_{U|10}(u)\ .
$$
Applying $F_{Y_{1}(0,0)|D=1,M=0}^{-1}(q)$ to both sides, we have
\begin{center}
$h(0,0,1,\ u)=F_{Y_{1}(0,0)|D=1,M=0}^{-1}(F_{U|10}(u))$ .   (1.2)
\end{center}\end{document}
