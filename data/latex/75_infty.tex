\documentclass[a4paper,12pt]{article}
\usepackage{latexsym}
\usepackage{amsmath}
\usepackage{amssymb}
\usepackage{graphicx}
\usepackage{wrapfig}
\pagestyle{plain}
\usepackage{fancybox}
\usepackage{bm}

\begin{document}

Here $A=\{\mathrm{t}\in \mathbb{R}^{\mathrm{d}}\ :\ \mathrm{d}_{\mathrm{G}\mathrm{Y}}(\mathrm{t},\ \mathrm{F},\ \mathrm{G})\ \leq \mathrm{d}_{\mathrm{G}\mathrm{Y}}\ (\ovalbox{\tt\small REJECT},\ \mathrm{F},\ \mathrm{G})\}$, where ? is any point

in $\mathbb{R}^{d}$ such that $\Delta$(?, $\mathrm{F}$) $=\ovalbox{\tt\small REJECT}$ and $\ovalbox{\tt\small REJECT}=G^{-1}$ (a) is a large quantile of $G$. Then,

we flag $\lfloor nd_{n}\rfloor$ observations. It is easy to see that,
$$
d_{n}=\sup_{\mathrm{t}\in \mathrm{A}}\{[1-\ H\ovalbox{\tt\small REJECT} n\ (\Delta\ (\mathrm{t},\ \mathrm{F}\ovalbox{\tt\small REJECT}\ \mathrm{n}))]-[1-\mathrm{G}(\Delta(\mathrm{t},\ \mathrm{F}\ovalbox{\tt\small REJECT}\ \mathrm{n}))]\}^{+}
$$
\begin{center}
$=\displaystyle \sup_{\mathrm{t}\in \mathrm{A}}\{G$ ($\Delta$ ($\mathrm{t},\ \mathrm{F}$ˆ $\mathrm{n}))- \mathrm{H}$ˆ $\mathrm{n}(\Delta(\mathrm{t},\ \mathrm{F}$ˆ $\mathrm{n}))\}^{+}$
$$
=\sup\{G(\Delta)-H\ovalbox{\tt\small REJECT}\ n(\Delta)\}^{+}
$$
\end{center}
$\Delta\geq$?

since $d_{GY}$ isanon increasing function of the squared Mahalanobis distance

of the point $\mathrm{t}$. We can rephrase Proposition 2. in ?, that states the con-

sistency property of the filter as follows. Consider a random vector $\mathrm{Y} =$

$(\mathrm{X}_{1},\ .\ .\ .\ ,\ \mathrm{X}_{\mathrm{d}}) \sim \mathrm{F}_{0}$ and a pair of location and scatter estimators $\mathrm{T}_{0\mathrm{n}}$ and

$\mathrm{C}_{0\mathrm{n}}$ such that $\mathrm{T}_{0\mathrm{n}} \rightarrow\ovalbox{\tt\small REJECT} 0 =$ µ(F0) $\in \mathbb{R}^{\mathrm{d}}$ and $\mathrm{C}_{0\mathrm{n}} \rightarrow \Sigma_{0} = \Sigma(\mathrm{F}_{0})$ a.s..

Consider any continuous distribution function $G$ and let {\it Hn} be the empirical
$$
n
$$
distribution function of $\Delta_{i}$ and $H_{0}(t) =\mathrm{P}\mathrm{r}$( ($\mathrm{Y}$--µ0)t$\Sigma$0-1 (Y--µ0) $\leq \mathrm{t}$). If

the distribution $G$ satisfies:
\begin{center}
$\displaystyle \max_{\mathrm{t}\in \mathrm{A}}\{d_{GY}(\mathrm{t},\ \mathrm{F}_{0},\ \mathrm{H}_{0})-\mathrm{d}_{\mathrm{G}\mathrm{Y}}(\mathrm{t},\ \mathrm{F}_{0},\ \mathrm{G})\}\leq 0$,   (1)
\end{center}
where $A=\{\mathrm{t}\in \mathbb{R}^{\mathrm{d}}\ :\ \mathrm{d}_{\mathrm{G}\mathrm{Y}}(\mathrm{t},\ \mathrm{F}_{0},\ \mathrm{G})\ \leq \mathrm{d}_{\mathrm{G}\mathrm{Y}}\ (\ovalbox{\tt\small REJECT},\ \mathrm{F}_{0},\ \mathrm{G})\}$, where ? is any point

in $\mathbb{R}^{d}$ such that $\Delta$(?, $\mathrm{F}_{0}$) $=\ovalbox{\tt\small REJECT}$ and $\ovalbox{\tt\small REJECT}=G^{-1}$ (a) is a large quantile of $G$, then

$n_{0}n \rightarrow 0$ a.s.

where
$$
n_{0}=\ \lfloor nd_{n}\rfloor.
$$
{\it Proof}. Note that

$d_{GY}$ ($\mathrm{t}$, Fˆ $\mathrm{n}$, Hˆ $\mathrm{n}$) $-\mathrm{d}_{\mathrm{G}\mathrm{Y}}$($\mathrm{t}$, Fˆ $\mathrm{n}, \mathrm{G}$) $=\mathrm{G}(\Delta(\mathrm{t},\ \mathrm{T}_{0\mathrm{n}},\ \mathrm{C}_{0\mathrm{n}}))-$ Hˆ $\mathrm{n}(\Delta(\mathrm{t},\ \mathrm{T}_{0\mathrm{n}},\ \mathrm{C}_{0\mathrm{n}}))$

and condition in equation $()$ is equivalent to
\begin{center}
$\mathrm{m}\Delta\geq \mathrm{a}$?$\mathrm{x} \{G(\Delta)\ -H_{0}(\Delta)\}\leq 0,$
\end{center}
The rest of the proof is the same as in Proposition 2. of ?.
$$
\square 
$$
Figure shows the bivariate scatter plot of WTS versus HTLD, HTLD

versus WSBC and WSBC versus SUR where the GY-UBF and HS-UBF

filters are applied, respectively. The bivariate observations with at least

one component flagged as outlier are in blue, and outliers detected by the

bivariate filter are in orange. We see that the HS-UBF identifies less outliers

with respect to the GY-UBF.

1
\end{document}
