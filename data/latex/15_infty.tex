\documentclass[a4paper,12pt]{article}
\usepackage{latexsym}
\usepackage{amsmath}
\usepackage{amssymb}
\usepackage{graphicx}
\usepackage{wrapfig}
\pagestyle{plain}
\usepackage{fancybox}
\usepackage{bm}

\begin{document}

The non-zero components of the Ricci tensor are
$$
R_{uu}=-5(f''+(f')^{2})
$$
while the Ricci scalar is given by

$R_{mn}=-g_{mn}(f''+5(f')^{2})$ (1)
\begin{center}
$R=-10f''-30(f')^{2}$   (2)
\end{center}
Furthermore, we have that $G=e^{5f} g$, where $g$ is the determinant of the unit $S^{5}$ metric.

Upon integration by parts, part of the Einstein-Hilbert term cancels with the Gibbons-

Hawking term to give the following simple expression
\begin{center}
$S=\ du\ d^{5}x\ ge^{5f}[-5((f')^{2}+4W^{2})+2\mathcal{L}_{\mathrm{k}\mathrm{i}\mathrm{n}}]$   (3)
\end{center}
The restriction to the flat case was not strictly necessary so far, but it will be crucial in the

next step. The gradient flow equations, together with the chain-rule, allows us to rewrite
\begin{center}
$\mathcal{L}_{\mathrm{k}\mathrm{i}\mathrm{n}}=-2W'$   (4)
\end{center}
Plugging this into and using the BPS equation of the warp factor, we find
\begin{center}
$S=-4\ d^{5_{X}}\ ge^{5f}W|_{0}^{\Lambda}$   (5)
\end{center}
where $\Lambda$ is the UV cutoff. Only the $\Lambda$ part of the action contributes, since $e^{5f}W|_{0}$ vanishes

due to the close-0ff of the geometry. Removing the UV cutoff $\Lambda \rightarrow \infty$ is equivalent to

removing the cutoff $\epsilon$ on our asymptotic coordinate $z$, i.e. $\epsilon\rightarrow 0$. From the UV asymptotics

we find that in this limit the factor $e^{5f}$ diverges like
\begin{center}
$e^{5f}\displaystyle \sim\frac{1}{\epsilon^{5}}$   (6)
\end{center}
This is the reason for the previous claims that only the terms up to $O(z^{5})$ in the superpoten-

tial are relevant for obtaining counterterms. All the higher-order terms vanish as the cuto

is removed. We may thus legitimately insert the approximate superpotential into to get the

counterterms,
\begin{center}
$S_{\mathrm{c}\mathrm{t}}^{(W)}=4\displaystyle \ d^{5}x\ \gamma\ [\frac{1}{2}+\frac{3}{4}\sigma^{2}+\frac{1}{16}(\phi^{0})^{2}-\frac{3}{16}(\phi^{3})^{2}+\frac{1}{192}(\phi^{0})^{4}-\frac{3}{16}(\phi^{0})^{2}\sigma]$   (7)
\end{center}
where $\gamma$ is the induced metric on the $ z=\epsilon$ boundary. All fields are evaluated at $ z=\epsilon$. This

gives all finite and infinite counterterms for the flat domain wall solutions.

1
\end{document}
