\documentclass[a4paper,12pt]{article}
\usepackage{latexsym}
\usepackage{amsmath}
\usepackage{amssymb}
\usepackage{graphicx}
\usepackage{wrapfig}
\pagestyle{plain}
\usepackage{fancybox}
\usepackage{bm}

\begin{document}

1 Monte Carlo results

We performed a Monte Carlo simulation to assess the performance of the

proposed filter based on halfspace depth. After the filter flags the outlying

observations, the generalized $\mathrm{S}$-estimator is applied to the data with added

missing values. Our simulation study is based on the same setup described

in ? to compare significantly the performance of our filter with respect to

the filter introduced in their work. We considered samples from a $N_{p}(0,\ \Sigma_{0})$ ,

where all values in {\it diag} $(\Sigma_{0})$ are equal to 1, $p = 10$, 20, 30, 40, 50 and the

sample size is $n=10p$. We consider the following scenarios:

$\bullet$ Clean data: data without changes.

$\bullet$ Cell-Wise contamination: a proportion $\epsilon$ of cells in the data is replaced

by $X_{ij}\sim N(k,\ 0.1^{2})$ , where $k=1$, . . . , 10.

$\bullet$ Case-Wise contamination: aproportion $\epsilon$ of cases in the data matrix

is replaced by $\mathrm{X}_{\mathrm{i}} \sim 0.5\mathrm{N}(\mathrm{c}\mathrm{v},\ 0.1^{2}\mathrm{I})+0.5\mathrm{N}(-\mathrm{c}\mathrm{v},\ 0.1^{2}\mathrm{I})$ , where $c=$

$\ovalbox{\tt\small REJECT}$ {\it k}(?{\it p}2)-1(0.99), $k=1$, 2, . . . , 20 and $\mathrm{v}$ is the eigenvector corresponding

to the smallest eigenvalue of $\Sigma_{0}$ with length such that (v-µ0)T $\Sigma_{0}^{-1}(\mathrm{v}-$
$$
\ovalbox{\tt\small REJECT} 0)=1.
$$
The proportions of contaminated rows chosen for case-wise contamination are

$\epsilon= 0.1$, 0.2, and $\epsilon= 0.02$, 0.05 for cell-wise contamination. The number of

replicates in our simulation study is $N=200$. We measure the performance

of a given pair of location and scatter estimators µˆ and $\Sigma^{}$ using the mean

squared error (MSE) and the likelihood ratio test distance (LRT), as in:
\begin{center}
$MSE=\displaystyle \frac{1}{N}\sum_{i=1}^{N}$ (µˆ$i$ --µ0)T (µˆi --µ0)

$LRT$ ( $\Sigma^{},\ \Sigma_{0}$) $=\displaystyle \frac{1}{\mathrm{N}}\sum_{\mathrm{i}=1}^{\mathrm{N}}\mathrm{D} (\Sigma^{} \mathrm{i}, \Sigma_{0})$
\end{center}
2 Statistical data depth properties

A depth function $d(\cdot;F)$ measures the centrality of a point w.r. $\mathrm{t}$. aprob-

ability distribution $F.$
$$
d=\mathbb{R}^{p}\rightarrow \mathbb{R}^{+}\cup\{0\},\ \mathrm{x}\rightarrow \mathrm{d}(\mathrm{x};\mathrm{F})
$$
A statistical depth function should satisfy the following Properties

1
\end{document}
